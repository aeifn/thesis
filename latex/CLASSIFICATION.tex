\chapter{Приложения}

\section*{Определения теории игр}

Теория игр представляет собой раздел математики, в котором исследуются модели принятия оптимальных решений в условиях конфликта \cite{mazalov2017}


есть различные равнове- сия: равновесие по Нэшу, равновесие по Штакельбергу, равновесие по Вардропу\cite[4]{mazalov2017}

\section{Место задачи в теории игр}

Математическая теория игр является составной частью исследования операция. \cite{petrosyan}

Теория игр --- раздел математики, изучающий математические модели принятия решений в условиях конфликта, когда каждый участник действует в своих интересах. \cite[7]{petrosyan}

Уровни информированности о ситуации --- детерменированный, стохастический и неопределенный.

Задача зачастую сводится к нахождения экстремума функции или ее матожидания.

По определению Н. Н. Воробьева \footnote{Воробъев Н. Н. Философская энциклопедия. Т. 5. М., 1970. С. 208—210.}, теория игр --- теория математических моделей принятия решений в условиях неопределенности, когда принимающий решение игрок располагает информацией лишь о множестве возможных ситуаций, в одной из которых он в действительности находится.




\section{Основные понятия}

\subsection*{Классификация теории игр\footnote{http://ru.discrete-mathematics.org/fall2019/elective/ilinskii.pdf}\cite[223]{Association:2018aa}}

Кооперативные и некооперативные.


Симметричный и несимметричные 

По выигрышу
Антагонистические (с нулевой суммой) и ненулевой суммой

Параллельные и последовательные

С полной и неполной информацией


По форме представления
В нормальной форме (в виде платежной матрицы --- игроки получают информацию сразу)
В развернутой форме (в виде дерева --- игроки получают получают информацию по мере игры)

По количеству стратегий
На конечны

Это важное приложение — необходимо понимать место рассматриваемой задачи в теории игр, обозреть и закрепить весь предмет, изучаемый в двухгодичном магистерском курсе.

Обзор дается по книгам \cite{nisan}
В работе уделено внимание классификации теории игр, чтобы было понимание, какое место задача занимает в отрасли знания.

Выбор игрока влияет на других игроков.

Кооперативная теория игр
Некооперативная теория игр
Игры в нормальной форме
Доминантные и доминирующие стратегии
Минимакс теорема
Игры в развернутой форме
Равновесие совершенное по подыграм
Повторяющиеся игры
Глобальные игры (Global games)
Игры в стратегической форме

% Выписать структуры теории игр
