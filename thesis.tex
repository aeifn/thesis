\documentclass{article}

\author{Егор Кузьмичев}

\begin{document}

ABSTRACT


«Теоретико-игровой анализ транспортных пробок вокруг мест проведения массовых мероприятий»



\section{Введение}

В работе уделено внимание классификации теории игр, чтобы было понимание, какое место задача занимает в отрасли знания.

\section{Классификация теории игр}

Выбор игрока влияет на других игроков.



Кооперативная теория игр

Некооперативная теория игр
Игры в нормальной форме
Доминантные и доминирующие стратегии
Минимакс теорема
Игры в развернутой форме
Равновесие совершенное по подыграм
Повторяющиеся игры

Глобальные игры (Global games)

Игры в стратегической форме


\section{Постановка задачи}

\section{Место задачи в классификации теории игр}

\section{Актуальность задачи}

\section{Новизна задачи}

\section{Библиография}

Aviad Heifetz Game Theory Interactive Strategies in Economics and ManagementISBN 978-0-521-76449-0 Cambridge University Press 2012

\end{document}
