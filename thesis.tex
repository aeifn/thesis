\documentclass[]{report}

\usepackage{cmap}
\usepackage[russian]{babel}
\usepackage[utf8]{inputenc}
\usepackage[T2A]{fontenc}

\author{Егор Кузьмичев}
\title{Теоретико-игровой анализ транспортных пробок вокруг мест проведения массовых мероприятий}
%\contentsname{Содержание}

\begin{document}
\maketitle

ABSTRACT

\tableofcontents


\section{Введение}

В работе уделено внимание классификации теории игр, чтобы было понимание, какое место задача занимает в отрасли знания.

\section{Классификация теории игр}

Выбор игрока влияет на других игроков.



Кооперативная теория игр

Некооперативная теория игр
Игры в нормальной форме
Доминантные и доминирующие стратегии
Минимакс теорема
Игры в развернутой форме
Равновесие совершенное по подыграм
Повторяющиеся игры

Глобальные игры (Global games)

Игры в стратегической форме


\section{Постановка задачи}

\section{Место задачи в классификации теории игр}

\section{Актуальность задачи}

\section{Новизна задачи}



\end{document}
